We propose the following algorithmic approaches to find the least
sensitive point (Definition~\ref{def:lsp}):
\begin{itemize}
\item Bisection Search
\item Basic Genetic Algorithm
\item Genetic Algorithm with Spectral Embeddings
\end{itemize}
We empirically evluate all three mehods in a case study in Section~\ref{sec:case_study}.

\paragraph*{Bisection Search}
The Bisection Search is based on the fact that for the least sensitive point
there has to exist some leak area $\alpha^*$, such that a leak of that area
remains undetected only at the LSP, while a leak of the same area would be
detected at every other node. Thus, one can perform a simple line search over
the leak area in order to find $\alpha^*$. Bisection search can be costly, in
particular when all timesteps in the simulation are viewed as potential
starting times of the leak. However, it yields the benefit that it is
guaranteed to find the least sensitive point in the given search space. To
reduce the computational load, we implement a pruning of nodes and starting
times based on intermediate results. Let $\alpha^i$ be the leak area after
iteration $i$. If a leak of size $\alpha^i$ remained undetected for at least
one node-time pair, we remove the following elements from the search space:
\begin{enumerate}
\item Every node for which a leak of size $\alpha^i$ was detected at every
starting time
\item Every starting time for which a leak of size $\alpha^i$ was detected at
every node
\end{enumerate}
In practise, this lead to a strong reduction of the search space after the
first iteration.

\paragraph*{Genetic Algorithm}
In the Basic Genetic Algorithm, we encode nodes and leak starting time as genes
and optimize the following fitness function
\begin{equation}
\Fitness (v_n,t) = \max \alpha \quad \text{s.t.} \fdetect(\fmeasure(\alpha
\V{e}_n, t)) = 0
\end{equation}
The largest undetected leak for each node-time pair is again computed via
a line search. In order to avoid unnecessary area maximization, we use
dynamic programming: The largest undetected leak area found so far is tested
first for every new node-time pair. If this is detected, no area maximization
was performed and the fitness is set to zero.

\paragraph*{Genetic Algorithm with Spectral Embeddings}
In order to take the network topology into account, we consider a second
version of the Genetic Algorithm, using a spectral analysis of the network's
graph Laplacian. Every node is assigned a four dimensional vector, containing
elements of the 2nd through 5th eigenvector of the graph Laplacian for that
node. Each vector element of the node embedding is then treated as a
different gene by the Genetic Algorithm. After every recombination and
mutation step, the nearest neighbour of the resulting embedding vector is
returned as offspring.\\

None of the approaches used here makes any assumptions on the type of leakage
detector. Hence, they are all model-agnostic in terms of the taxonomy (Figure
~\ref{fig:adversarial_taxonomy}).
