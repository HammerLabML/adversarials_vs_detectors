In this work, we applied the concept of adversarial attacks to leakage
detection in water distribution networks. For this purpose, we proposed a
taxonomy of potential adversarials against a leakage detector. We then focused
on a particular type of adversarial: Attacking the network at the least
sensitive point. We formalized the least sensitive point
problem and proposed three algorithmic approaches to solve it.
In practice, knowing the least sensitive point and the vulnerability to attacks constitutes crucial information which enables practitioners to develop more robust methods using e.g. adversarial training~\cite{bai2021recent} or an improvement of the robustness by introducing targeted sensors.

We empirically evaluated our proposed methods in a case study on two benchmark WDNs.
For the genetic algorithm with spectral embeddings, experiments yielded
promising results on both networks. This method allows much faster solutions
when compared to the Bisection Search, due to a lower runtime complexity. The
locations of the least sensitive point indicate that the detectors weak spots
are highly dependent on network topology, in particular on the location of the
water sources.

In future work, we plan to address the same problem with another leakage
detection model to compare results between different leakage detectors.
In order to increase the robustness, we will determine the
effect of incremental sensor placement on the maximum undetected leak area.
For this purpose, we will simulate sensors at the location where the least
sensitive point has previously been detected and re-run the experiments with
the new sensors in place. In this way, knowledge about adversarials can be 
helpful in practical applications.
