Leakage detectors typically rely on measurements of hydraulic variables such
as pressure and flow at certain points in a network. Measurements are taken at
discrete timesteps and depend on various factors, including network topology,
water consumption at network nodes and potential leakages. When creating an
adversarial input to the detector, we are mainly interested in how the
pressure values are influenced by leaks. Hence, we consider measurements based
on leakage events characterized by the area of the leak. All other factors
like water consumption and network topology are assumed to be fixed.

\begin{definition}[Measurement Function]
A measurement function maps a vector of leak areas $\V{a}_t \in \mathbb{R}^N$ for all $N$ nodes of a water distribution network at some timestep $t$ to pressure measurements at some nodes equipped with sensors. It is parametrized also by the timestep itself.
\begin{equation}
\begin{split}
\fmeasure: \mathbb{R}^N \times \{ 1, \hdots, T \} &\to \mathbb{R}^S\\
(\V{a}_t, t) &\mapsto \V{y}_t
\end{split}
\end{equation}
\end{definition}

Given the pressure measurements, we can now define the leakage detector as a
binary predictor

\begin{definition}[Leakage Detector]
A leakage detector is a function mapping pressure measurements to a binary output, indicating whether a leak was detected or not.
\begin{equation}
\begin{split}
\fdetect: \mathbb{R}^S &\to \{ 0, 1 \}\\
\V{y}_t &\mapsto \begin{cases*}
1 & \text{if a leak was detected}\\
0 & \text{otherwise}
\end{cases*}
\end{split}
\end{equation}
\end{definition}
